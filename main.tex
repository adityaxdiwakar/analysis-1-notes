\documentclass[14pt]{extarticle}
\usepackage[utf8]{inputenc}
\usepackage{amsmath}
\usepackage{amsthm}
\usepackage{amssymb}
\usepackage[lmargin=1.5in, rmargin=1.5in, bmargin=1in]{geometry}
\usepackage{enumitem}
\usepackage{etoc}
\usepackage{mathtools}
\usepackage{fancyhdr}
\usepackage{graphicx}
\usepackage[english]{babel}
\pagestyle{fancy}
\fancyhf{}
\fancyhead[R]{\thepage}

\theoremstyle{definition}
\newtheorem{definition}{Definition}[section]
\newtheorem{axiom}{Axiom}[section]
\newtheorem{theorem}{Theorem}[section]
\newtheorem{example}{Example}[section]

\renewcommand*\contentsname{Table of Contents}
\renewcommand{\headrulewidth}{0pt}
\graphicspath{ {./img/} }
\setcounter{tocdepth}{1}
\newcommand{\Mod}[1]{\ (\mathrm{mod}\ #1)}


\etocsettocstyle{}{}

\begin{document}

\begin{titlepage}
   \begin{center}
       \vspace*{1cm}

       \textbf{\LARGE Real Analysis I}

       \vspace{0.5cm}

       {\Large ...something something real numbers}
            
       \vspace{1.5cm}

       \textbf{\Large Aditya Diwakar}

       \vfill
            
       \large{In hopes to not fail MATH 4317\dots
            
       \vspace{0.8cm}
     
       Georgia Institute of Technnology\\
       May 13th, 2021}
            
   \end{center}
\end{titlepage}

\section{Introduction}
This is a collection of notes taken while self studying Real Analysis from the book Principles of 
Mathematical Analysis. It is meant for personal use, but it is open sourced so that others can benefit 
as well. In the case there is an issue with these notes, please write an email to aditya@diwakar.io. 
A table of contents is presented below.

\tableofcontents
\pagebreak
\setcounter{tocdepth}{3}

\section{Ordered Sets}
Ordered sets are any sets that have a definition regarding order in them. Essentially, if there is some
well-defined relation that determines which element precedes any other element.

\begin{definition}
    Suppose $S$ is an ordered set and we take $E \subset S$. 
    \begin{enumerate}[label=(\arabic*)]
        \item If there exists some element $\alpha\in S$ such that
        $\alpha \geq e$ for every $e\in E$, then we say that $E$ is bounded from above (and it is bounded by $\alpha$).

    \item If there exists some element $\beta \in S$ such that $\beta \leq e$ for every $e\in E$, then we say $E$ is bounded from below
        (and it is bounded beta $\beta$)
    \end{enumerate}

\end{definition}

Obviously, there can be more than one bounding number as it is not required that the bound exists in the set it bounds. In fact, if
there exists an upper or lower bound on a set, then there will always be a number greater or a number smaller that is also a bound (on 
either bound). To combat this lack of specificity, we define the following:

\begin{definition}
    Suppose we have $S$ which is an ordered set and have $E\subset S$ that is bounded. Let us say there exists some $\alpha, \beta \in S$
    where $\alpha$ is an upper bound and $\beta$ is a lower bound.
    \begin{enumerate}[label=(\arabic*)]
        \item If $\gamma < \alpha \implies \gamma$ is not an upper bound for every $\gamma$, then $\alpha$ is the least upper bound 
            (supremum).
            \begin{align*}
                \alpha = \sup E
            \end{align*}
        \item If $\gamma > \beta \implies \gamma$ is not a lower bound for every $\gamma$, then $\alpha$ is the greatest lower bound 
            (infimum).
            \begin{align*}
                \beta = \inf E
            \end{align*}
    \end{enumerate}
\end{definition}

It is clear that if $E\subset S$ where $E \neq \emptyset$ is bounded above, then $\exists\sup E \in S$ and similarly if bounded below,
then $\exists\inf E\in S$.

\pagebreak
\begin{theorem}
    Suppose $S$ is an ordered set and $E\subset S$ and $E$ is bounded below. Let $L$ be the set of all the possible lower bounds of $E$,
    then
    \begin{align*}
        \left(\alpha \in S\right) = \sup L = \inf E
    \end{align*}
    This means that the supremum of the set of lower bounds is the infimum of the subset itself.
\end{theorem}

\begin{proof}
    Since $E$ is bounded by below, we have $L\neq \emptyset$. Notice, that since $L$ is the set of lower bounds, every element $e\in E$
    is greater than every element $y\in L$. Hence, $L$ is bounded from above and $\exists\sup L\in S$, say $\alpha$.
    
    Next, notice that for any element $\gamma \in E$, we have that $\gamma \geq \alpha$ by definition. This is beacuse $\alpha$ is the
    greatest element in $L$ and $L$ is the set of lower bounds. In words, $\alpha$ is the greatest lower bound. By definition,
    $\alpha = \inf E$. 
\end{proof}

\pagebreak

\section{Finite and Countable Sets}
The notion of sets and the cardinality of those sets is obvious from prerequisite content. However, when we have an infinite set,
there are different characteristics that should be discussed. Not all infinity is the same. 

\begin{definition}
    Let there exist a set $J_n$ where $n\in \mathbb{Z}^+$. For any set $A$, we say:
    \begin{enumerate}[label=(\alph*)]
        \item A is finite if $A \sim J_n$ for some $n$ ($\sim$ is an equivelance relation)
        \item A is infinite if A is not finite
        \item A is countable if $A\sim J_n$ (for some $n$)
        \item A is uncountable if A is neither finite nor countable
        \item A is at most countable if A is finite or countable
    \end{enumerate}
\end{definition}

\noindent
Another word for countable sets is enumerable/denumerable sets.

\begin{example}
    How do we arrange the set of all integers as a countable set?

    \begin{proof} If we arrange the integers $A = \mathbb{Z}$ starting with 0 and then the positive
        integer before that integer's additive inverse, then we can see that $\mathbb{Z}$ is countable.
        \begin{align*}
            A:\quad &0, 1, -1, 2, -2, 3, -3\\
            J:\quad &1, 2, 3, 4, 5, 6, 7
        \end{align*}
    \end{proof}
    In general, this would mean that we can define some function that shows $\mathbb{Z}$ is countable, like so:
    \begin{align*}
        \begin{cases}
            0 & n = 0\\
            \frac{n}{2} & n\equiv 0 \Mod{2}\\
            \frac{-(n-1)}{2} & n\equiv 1 \Mod{2}
        \end{cases}
    \end{align*}
\end{example}

\pagebreak
\begin{theorem}
    Every infinite subset of a countable set $A$ is countable.

    \begin{proof}
        Since $A$ is countable and $E$ is an infinite subset, we can arrange $A$ in a way that makes it obvious that $A\sim J$. 
        With this in mind, we can say every element of $A$ has a certain index. From this, we define $n_k$ as the $k^{th}$ value 
        in $E$ (meaning $x_{n_k} \in E$). Let $x_{n_1}$ be the smallest value in $E$ and then define $n_2, n_3, ...$ as the next
        smallest values. From this, we now see that $k$ is the index for the values in $E$. In other words, we have $f(k) = x_{n_k}$
        showing that $E\sim J\implies E$ is countable.
    \end{proof}
\end{theorem}

\begin{definition}
    Let $A$ and $\Omega$ be sets and suppose that with each element $\alpha$ of $A$, there is an associated subset of $\Omega$
    denoted by $E_\alpha$. Further, the set of sets (family or collection of sets) for all the difference $\alpha\in A$ is denoted
    by $\{E_\alpha\}$
\end{definition}

\begin{theorem}
    Let $\{E_n\}$ where $n\in \mathbb{Z}^+$ be a sequence of countable sets (elements are sets) and let
    \begin{align*}
        S = \bigcup_{n=1}^\infty E_n
    \end{align*}
    then $S$ is countable ($S\sim J$).
\end{theorem}

\begin{theorem}
    The set of all rational numbers is countable.

    \begin{proof}
        At first glance, this does not look intuitive. However, if we construct $\mathbb{Q}$ as pairs of elements in $\mathbb{Z}$, then
        $\mathbb{Q}$ is simply defined as $(a, b)$ with $a, b\in \mathbb{Z}$. In fact, if $b = 1$, then $(a, 1)$ with $a\in \mathbb{Z}$
        is $\mathbb{Z}$ itself. We know that this set of pairs is countable, meaning that the rational numbers are countable.
    \end{proof}
\end{theorem}

Not every infinite set is countable. A quick trivial example of this would be a set with elements $0$ or $1$ which cannot be indexed, this
set is not countable (ex: $A = \{1, 0, 0, 1, 0, 1, 1, 1, ...\} \not\sim J$)

\pagebreak

\section{Fields}
A field (from abstract algebra) has two operations: addition and multiplication. Further, it also has multiplicative inverses
when the element is not zero.

\begin{axiom}
    Every field follows the following axioms: 
    \begin{enumerate}[label=(\alph*)]
        \item Axioms for addition
            \begin{enumerate}[label=(\arabic*)]
                \item $x, y\in F \implies x + y\in F$ 
                \item $\forall x,y\in F:\ x + y = y + x$
                \item $\forall x,y,z\in F:\ (x + y) + z = x + (y + z)$ 
                \item $\exists\ 0\in F:\ 0 + x = x,\ \forall x\in F$
                \item $\exists\ -x\in F\ \forall\ x\in F$ such that $-x + x = 0$
            \end{enumerate}
        \item Axioms for multiplication
            \begin{enumerate}[label=(\arabic*)]
                \item $x, y\in F \implies xy\in F$ 
                \item $\forall x,y\in F:\ xy = yx$
                \item $\forall x,y,z\in F:\ (xy)z = x(yz)$
                \item $\exists\ 1\in F:\ 1x = x,\ \forall x\in F$
                \item $\exists\ 1/x\in F\ \forall\ x\in F$ such that $x(1/x) = 1$
            \end{enumerate}
        \item Distributive laws
            \begin{align*}
                x(y + z) = xy + xz\quad \forall x,y,z\in F
            \end{align*}
    \end{enumerate}
\end{axiom}

There are numerous theorems that can be proved from here which define standard algebraic operations. Many of these are trivial and
will be omitted accordingly.

\begin{theorem}
    The field R which is ordered and has the least upper-bound property. 
\end{theorem}

Reminder, this property is if a bound exists on a nonempty subset, then $\sup E$ exists (if $E\subset R$)

\pagebreak

\begin{theorem} Archimedean property (a) and Q dense in R (b)
    \begin{enumerate}[label=(\alph*)]
        \item $x,y\in R$ with $x > 0 \implies \exists\ n\in \mathbb{Z}^+: nx > y$
        \item $x,y\in R$ with $x < y \implies \exists\ p\in \mathbb{Q}: x < p < y$
    \end{enumerate}

    The proof is simple, but not important. (a) is proved using a contradiction assuming that $y$ is the upper bound of the set, which
    is false. (b) is a bit more tricky, but essentially concludes that $\mathbb{Q}$ is dense in $\mathbb{R}$ and between any
    two irrational numbers exists a rational number (given by $p\in \mathbb{Q}$)
\end{theorem}

\begin{theorem}
    $\forall x>0\in R, n>0\in \mathbb{Z}$, there exists a unique $y\in R: y^n = x$
\end{theorem}

The proof of this theorem makes use of showing contradictions appear for when $y^n < x$ and $y^n > x$ leaving $y^n = x$ as the
only remaining option.
\end{document}
