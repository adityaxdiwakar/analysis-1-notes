\documentclass[14pt]{extarticle}
\usepackage[utf8]{inputenc}
\usepackage{amsmath}
\usepackage{amsthm}
\usepackage{amssymb}
\usepackage[lmargin=1.5in, rmargin=1.5in, bmargin=1in]{geometry}
\usepackage{enumitem}
\usepackage{etoc}
\usepackage{mathtools}
\usepackage{fancyhdr}
\usepackage{graphicx}
\usepackage[english]{babel}
\pagestyle{fancy}
\fancyhf{}
\fancyhead[R]{\thepage}

\theoremstyle{definition}
\newtheorem{definition}{Definition}[section]
\newtheorem{axiom}{Axiom}[section]
\newtheorem{theorem}{Theorem}[section]

\renewcommand*\contentsname{Table of Contents}
\renewcommand{\headrulewidth}{0pt}
\graphicspath{ {./img/} }
\setcounter{tocdepth}{1}

\etocsettocstyle{}{}

\begin{document}

\begin{titlepage}
   \begin{center}
       \vspace*{1cm}

       \textbf{\LARGE Real Analysis I}

       \vspace{0.5cm}

       {\Large ...something something real numbers}
            
       \vspace{1.5cm}

       \textbf{\Large Aditya Diwakar}

       \vfill
            
       \large{In hopes to not fail MATH 4317\dots
            
       \vspace{0.8cm}
     
       Georgia Institute of Technnology\\
       May 13th, 2021}
            
   \end{center}
\end{titlepage}

\section{Introduction}
This is a collection of notes taken while self studying Real Analysis from the book Principles of 
Mathematical Analysis. It is meant for personal use, but it is open sourced so that others can benefit 
as well. In the case there is an issue with these notes, please write an email to aditya@diwakar.io. 
A table of contents is presented below.

\tableofcontents
\pagebreak
\setcounter{tocdepth}{3}

\section{Ordered Sets}
Ordered sets are any sets that have a definition regarding order in them. Essentially, if there is some
well-defined relation that determines which element precedes any other element.

\begin{definition}
    Suppose $S$ is an ordered set and we take $E \subset S$. 
    \begin{enumerate}[label=(\arabic*)]
        \item If there exists some element $\alpha\in S$ such that
        $\alpha \geq e$ for every $e\in E$, then we say that $E$ is bounded from above (and it is bounded by $\alpha$).

    \item If there exists some element $\beta \in S$ such that $\beta \leq e$ for every $e\in E$, then we say $E$ is bounded from below
        (and it is bounded beta $\beta$)
    \end{enumerate}

\end{definition}

Obviously, there can be more than one bounding number as it is not required that the bound exists in the set it bounds. In fact, if
there exists an upper or lower bound on a set, then there will always be a number greater or a number smaller that is also a bound (on 
either bound). To combat this lack of specificity, we define the following:

\begin{definition}
    Suppose we have $S$ which is an ordered set and have $E\subset S$ that is bounded. Let us say there exists some $\alpha, \beta \in S$
    where $\alpha$ is an upper bound and $\beta$ is a lower bound.
    \begin{enumerate}[label=(\arabic*)]
        \item If $\gamma < \alpha \implies \gamma$ is not an upper bound for every $\gamma$, then $\alpha$ is the least upper bound 
            (supremum).
            \begin{align*}
                \alpha = \sup E
            \end{align*}
        \item If $\gamma > \beta \implies \gamma$ is not a lower bound for every $\gamma$, then $\alpha$ is the greatest lower bound 
            (infimum).
            \begin{align*}
                \beta = \inf E
            \end{align*}
    \end{enumerate}
\end{definition}

It is clear that if $E\subset S$ where $E \neq \emptyset$ is bounded above, then $\exists\sup E \in S$ and similarly if bounded below,
then $\exists\inf E\in S$.

\pagebreak
\begin{theorem}
    Suppose $S$ is an ordered set and $E\subset S$ and $E$ is bounded below. Let $L$ be the set of all the possible lower bounds of $E$,
    then
    \begin{align*}
        \left(\alpha \in S\right) = \sup L = \inf E
    \end{align*}
    This means that the supremum of the set of lower bounds is the infimum of the subset itself.
\end{theorem}

\begin{proof}
    Since $E$ is bounded by below, we have $L\neq \emptyset$. Notice, that since $L$ is the set of lower bounds, every element $e\in E$
    is greater than every element $y\in L$. Hence, $L$ is bounded from above and $\exists\sup L\in S$, say $\alpha$.
    
    Next, notice that for any element $\gamma \in E$, we have that $\gamma \geq \alpha$ by definition. This is beacuse $\alpha$ is the
    greatest element in $L$ and $L$ is the set of lower bounds. In words, $\alpha$ is the greatest lower bound. By definition,
    $\alpha = \inf E$. 
\end{proof}

\end{document}

