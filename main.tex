\documentclass[14pt]{extarticle}
\usepackage[utf8]{inputenc}
\usepackage{amsmath}
\usepackage{amssymb}
\usepackage[lmargin=1.5in, rmargin=1.5in, bmargin=0.75in]{geometry}
\usepackage{enumitem}
\usepackage{etoc}
\usepackage{mathtools}
\usepackage{fancyhdr}
\usepackage{graphicx}
\usepackage[english]{babel}
\pagestyle{fancy}
\fancyhf{}
\fancyhead[R]{\thepage}

\theoremstyle{definition}
\theoremstyle{axiom}
\newtheorem{definition}{Definition}[subsection]
\newtheorem{axiom}{Axiom}[section]

\renewcommand*\contentsname{Table of Contents}
\renewcommand{\headrulewidth}{0pt}
\graphicspath{ {./img/} }
\setcounter{tocdepth}{1}

\etocsettocstyle{}{}

\begin{document}

\begin{titlepage}
   \begin{center}
       \vspace*{1cm}

       \textbf{\LARGE Real Analysis I}

       \vspace{0.5cm}

       {\Large ...something something real numbers}
            
       \vspace{1.5cm}

       \textbf{\Large Aditya Diwakar}

       \vfill
            
       \large{In hopes to not fail MATH 4317\dots
            
       \vspace{0.8cm}
     
       Georgia Institute of Technnology\\
       May 13th, 2021}
            
   \end{center}
\end{titlepage}

\section{Introduction}
This is a collection of notes taken while self studying Real Analysis from Terrence Tao's Analysis book.
It is meant for personal use, but it is open sourced so that others can benefit as well. In the case
there is an issue with these notes, please write an email to aditya@diwakar.io. A table of contents is
presented below.

\tableofcontents
\pagebreak
\setcounter{tocdepth}{3}

\section{Natural Numbers}
\localtableofcontents

\vspace{0.5cm}

\noindent
The most basic of numerical structures, the natural numbers are the smallest subset of infinite numbers
contained in every other set we will discuss.

\subsection{The Peano Axioms}
\begin{definition}
    A natural number is any number in the set given by
    \begin{align*}
        N \coloneqq \{0, 1, 2, 3, 4, \dots\}
    \end{align*}
    and the entire set is known as the natural numbers, denoted $\mathbb{N}$.
\end{definition}

For the sake of this class, the natural numbers excluding $0$, given by $\mathbb{N}\setminus \{0\}$
is denoted by $\mathbb{Z}^\plus$ (positive integers).

\begin{axiom}
    0 is a natural number and if $n$ is a natural number, this implies that $n++$ is also a natural number.
    This iteratively constructs the set $\mathbb{Z}$. 
\end{axiom}

\begin{axiom}
    Further, it is not possible to reach $0$ from any other natural number. In other words, 
    $0$ is the successor of no natural number.
\end{axiom}

There is another issue with this definition and it is given by the fact that this does not 
definitievely construct the natural numbers as $0 \to 1 \to 2 \to 2 \to 2 \to 2 \to \cdots$
is a valid number system by this design. We want to limit this from happening by introducing:

\begin{axiom}
    If two natural numbers are different, then their successors are also different. In other words,
    $m \neq n \implies m\tiny{++} \neq n\tiny{++}$.
\end{axiom}

\vspace{1cm}

\end{document}

